\documentclass[main.tex]{subfiles}

% Header from main.tex

\begin{document}
\section{Literature Review}
The dataset\cite{Irvine} has already been examined in two papers\cite{datamining}\cite{rulebased} that we examined so that we would not merely reproduce their results, but be able to learn from and add to.\\
In "A rule-based model for Seoul Bike sharing demand prediction using weather data" \cite{rulebased}, the authors collected the data from several APIs and merged them together to get a more complete view of biking activity. The paper then tried five approaches: CUBIST, regularized random forests, classification and regression trees and KNN to understand the dataset that they gathered. Of the five methods that they used to attempt to model the dataset, CUBIST showed the most promising results with an $R^2$ of 0.95 on the testing dataset, as well as lower root mean square errors, mean absolute error, and coefficent of variation than the other four methods presented.
In "Using data mining techniques for bike sharing demand prediction in metropolitan city" \cite{datamining} the authors used the same dataset with all of its added features to determine what was the smallest subset of the data that they could use and with which models could it be used. In the article, the authors explored linear regression, gradient boosting machines, support vector machines, boosted trees, and extreme gradient boosted trees. They found with all of the variables they were able to achieve an $R^2$ of 0.92 on the testing dataset using gradient boosted machines.
\end{document}
